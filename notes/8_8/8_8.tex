\documentclass{article}
\usepackage[margin=0.75in]{geometry}
\usepackage{amsmath}
\usepackage{amsfonts}
\usepackage{tikz}
\usepackage{xcolor}
\usepackage{hyperref}

\begin{document}

\title{
Improper Integrals
\\
\large
Calculus II \S 8.8
} 
\author{
Ian Turner
\\
\small
(\href{mailto:irturner@mailbox.sc.edu}{irturner@mailbox.sc.edu})
}
\maketitle

\section*{Main Ideas}

\begin{itemize}

\item
Improper integrals are definite integrals with infinities or asymptotes within their bounds.

\item
This can be dealt with by replacing the infinity or asymptote with a limit.

\item
If the limit does not converge, the integral does not exist.

\end{itemize}

\section*{Examples}

\begin{itemize}

\item
This integral has a infinite bounds

\begin{gather*}
\int_1^{\infty} \frac{1}{x^2} dx
\\
\\
\text{we can fix this by adding a limit}
\\
=\lim_{b \to \infty} \int_1^b \frac{1}{x^2} dx
\\
\\
\text{and the integral will exist if the limit exists}
\\
=\lim_{b \to \infty} \left[ \frac{-1}{x} \right]_1^b
=\lim_{b \to \infty} \left[ \frac{-1}{b} - (-1) \right]
\\
\\
=\frac{-1}{\infty}+1
=0+1
=1
\end{gather*}
\\

\item
This function goes to infinity within the integral's bounds

\begin{gather*}
\int_0^1 \frac{dx}{\sqrt{x}}
\\
\\
\text{we can also add a limit here}
\\
=\lim_{a \to 0} \int_a^1 \frac{dx}{\sqrt{x}}
=\lim_{a \to 0} \left[ ~ 2\sqrt{x} ~ \right]_a^1
\\
=\lim_{a \to 0} \left[ ~ 2\sqrt{1} - 2\sqrt{a} ~ \right]
=2
\end{gather*}

\end{itemize}

\newpage

\end{document}