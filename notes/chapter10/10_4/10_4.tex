\documentclass{article}
\usepackage[margin=0.75in]{geometry}
\usepackage{amsmath}
\usepackage{amsfonts}
\usepackage{tikz}
\usepackage{xcolor}
\usepackage{hyperref}
\usepackage{tabularx}
\usepackage{multicol}

\def\chapt{10.4}

\begin{document}

\noindent
{\scshape Math 142} \hfill {\scshape \S\chapt~ notes} \hfill {\scshape Spring 2024}

\smallskip

\hrule

\bigskip

\hfill
\\

\thispagestyle{empty}

{
\huge
\noindent
\textbf{10.4 Comparison Tests}
}

\section*{Main Ideas}

\begin{itemize}

% 
%
% Direct comparison test
%
%
\item
\textbf{Direct Comparison Test} (page 568)

\begin{gather*}
\sum a_n
\text{ and }
\sum b_n
\text{ are two series where }
0 \leq a_n \leq b_n
\text{ for all }
n
\end{gather*}

\begin{enumerate}
\item
If $\sum b_n$ converges, then $\sum a_n$ also converges.
\item
If $\sum a_n$ diverges, then $\sum b_n$ also diverges.
\end{enumerate}
\hfill
%
%
% Limit comparison test
%
%
\item
\textbf{Limit Comparison Test} (page 569)

\begin{gather*}
\text{If }
a_n > 0
\text{ and }
b_n > 0
\text{ for all }
n \geq N
\text{ where $N$ is some integer }
\end{gather*}

\begin{enumerate}
\item
If
\begin{gather*}
\lim_{n\to\infty} \frac{a_n}{b_n}=c \text{ and } c>0
\text{ then }
\sum a_n
\text{ and }
\sum b_n
\text{ both converge or both diverge.}
\end{gather*}
\item

If
\begin{gather*}
\lim_{n\to\infty} \frac{a_n}{b_n}=0
\text{ and }
\sum b_n
\text{ converges, then }
\sum a_n
\text{ converges.}
\end{gather*}
\item

If
\begin{gather*}
\lim_{n\to\infty} \frac{a_n}{b_n}=\infty
\text{ and }
\sum b_n
\text{ diverges, then }
\sum a_n
\text{ diverges.}
\end{gather*}
\end{enumerate}

\end{itemize}

\newpage

%
%
% Homework examples
%
%

\section*{Homework Problems}


\noindent
\textbf{
1.
}
\\
\begin{gather*}
\sum_{n=1}^{\infty} \frac{1}{n^2+30}
\\
\\
\text { is similar to the $p$-series }
\sum_{n=1}^{\infty} \frac{1}{n^2}
\\
\\
\implies \frac{1}{n^2+30}<\frac{1}{n^2}
\text{ for all } n \geq 1
\\
\\
\text{ since the $p$-series with $p=2$ converges, so does our series by direct comparison}
\end{gather*}



\noindent
\textbf{
3.
}
\\
\begin{gather*}
\sum_{n=1}^{\infty} \frac{1}{\sqrt{n}-1}
\\
\\
\text { is similar to the $p$-series }
\sum_{n=1}^{\infty} \frac{1}{\sqrt{n}}=\sum_{n=1}^{\infty} \frac{1}{n^{1/2}}
\\
\\
\implies  \frac{1}{\sqrt{n}-1}>\frac{1}{\sqrt{n}}
\text{ for all n }
\\
\\
\text{ since the $p$-series with $p=1/2$ diverges, so does our series by direct comparison}
\end{gather*}



\noindent
\textbf{
5.
}
\\
\begin{gather*}
\sum_{n=1}^{\infty} \frac{\cos^2 n}{n^{3/2}}
\\
\\
\text{ since $\cos^2$ is always between 0 and 1, }
\\
0 \leq \frac{\cos^2 n}{n^{3/2}} \leq \frac{1}{n^{3/2}}
\\
\\
\text{ since $\frac{1}{n^{3/2}}$ is a $p$-series with $p=3/2>1$, meaning it converges,}
\\
\text{
then by the direct comparison test our series also converges}
\end{gather*}



\newpage



\noindent
\textbf{
9.
}
\\
\begin{gather*}
\sum_{n=1}^{\infty} \frac{n-2}{n^3-n^2+3}
\\
\\
\text{compare with } 1/n^2
~~~~
\lim_{n \to \infty} \frac{\frac{1}{n^2}}{\frac{n-2}{n^3-n^2+3}}
\\
\\
=\lim_{n \to \infty} \frac{n^3-n^2+3}{n^2(n-2)}
\\
\\
=\lim_{n \to \infty} \frac{n^3-n^2+3}{n^3-2n^2}=1
\\
\\
\implies
\text{converges since $1/n^2$ converges}
\end{gather*}



\noindent
\textbf{
11.
}
\\
\begin{gather*}
\sum_{n=2}^{\infty} \frac{n(n+1)}{(n^2+1)(n-1)}
~~~\text{compare with}~~~
\sum_{n=2}^{\infty} \frac{1}{n-1}
\\
\\
\implies
\lim_{n \to \infty} \frac{\frac{n(n+1)}{(n^2+1)(n-1)}}{\frac{1}{n-1}}
=\lim_{n \to \infty} \frac{n(n+1)(n-1)}{(n^2+1)(n-1)}
=\lim_{n \to \infty} \frac{n(n+1)}{(n^2+1)}=1 \implies \text{both converge or diverge}
\\
\\
\sum_{n=2}^{\infty} \frac{1}{n-1}
=\sum_{n=1}^{\infty} \frac{1}{n}
~~~\text{($p$-series with $p=1$)} \implies \text{both \underline{diverge} by limit comparison}
\end{gather*}



\noindent
\textbf{
13.
}
\\
\begin{gather*}
\sum_{n=1}^{\infty} \frac{5^n}{\sqrt{n}~4^n}
\\
\\
\text{ compare with series }
\sum_{n=1}^{\infty} \frac{1}{\sqrt{n}}
\\
\\
\implies
\lim_{n \to \infty} \frac{\frac{5^n}{\sqrt{n}~4^n}}{\frac{1}{\sqrt{n}}}
= \lim_{n \to \infty} \frac{5^n}{4^n}= \lim_{n \to \infty}(5/4)^n=\infty
\\
\\
\text{since the series on the bottom diverges ($p$-series where $p=1/2\leq1$)}
\\
\text{the series on top diverges also by limit comparison}
\end{gather*}



\noindent
\textbf{
19.
}
\\
\begin{gather*}
\sum_{n=1}^{\infty} \frac{\sin^2 n}{2^n}
~~~\text{is bounded between 0 and } \frac{1}{2^n}
\\
\frac{1}{2^n} \text{ is a geometric series with } a=1, r=1/2<1
\implies \text{\underline{converges}}
\end{gather*}



\noindent
\textbf{
21.
}
\\
\begin{gather*}
\sum_{n=1}^{\infty} \frac{2n}{3n-1}
\\
\\
\lim_{n \to \infty} \frac{2n}{3n-1}=2/3 \neq 0
\implies \text{\underline{diverges} by $n$-th term test}
\end{gather*}



\noindent
\textbf{
33.
}
\\
\begin{gather*}
\sum_{n=2}^{\infty} \frac{1}{n\sqrt{n^2-1}}
\\
\\
=\sum_{n=1}^{\infty} \frac{1}{(n+1)\sqrt{(n+1)^2-1}}
=\sum_{n=1}^{\infty} \frac{1}{(n+1)\sqrt{n^2+2n}}
\\
\\
\text{ after change of bounds (start with $n=1$) is similar to the $p$-series }
\sum_{n=1}^{\infty} \frac{1}{n^2}
\\
\\
\implies
\frac{1}{(n+1)\sqrt{n^2+2n}}<\frac{1}{n^2}
\text{ for all } n \geq 1
\\
\\
\text{ and since $p=2>1$ means the $p$-series converges, our series also converges
by direct comparison}
\end{gather*}


\end{document}