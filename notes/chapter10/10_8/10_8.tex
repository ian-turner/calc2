\documentclass{article}
\usepackage[margin=0.75in]{geometry}
\usepackage{amsmath}
\usepackage{amsfonts}
\usepackage{tikz}
\usepackage{xcolor}
\usepackage{hyperref}
\usepackage{tabularx}
\usepackage{multicol}

\def\chapt{10.8}
\def\chaptname{Taylor and Maclaurin Series}

\begin{document}

\noindent
{\scshape Math 142} \hfill {\scshape \S\chapt~ notes} \hfill {\scshape Spring 2024}

\smallskip

\hrule

\bigskip

\hfill
\\

{
\huge
\noindent
\textbf{\chapt~\chaptname}
}

\section*{Main Ideas}

\begin{itemize}

\item
\textbf{Taylor Series}

If $f$ is a function with derivatives of all orders throughout some interval containing
$a$, then the \underline{Taylor series} generated by $f$ at $x=a$ is
\begin{gather*}
\sum_{n=0}^{\infty} \frac{f^{(n)}(a)}{n!}(x-a)^n
~~~=~~~
f(a)+f'(a)(x-a)+\frac{f''(a)}{2!}(x-a)^2+\frac{f^{(3)}(a)}{3!}(x-a)^3+...
\end{gather*}
where $f^{(n)}(x)$ is the $n$-th derivative of $f(x)$
\\

\item
\textbf{Maclaurin Series}
The \underline{Maclaurin} series of $f$ is the Taylor series of $f$ at $a=0$
\\

\item
A \textbf{Taylor Polynomial of Order $n$} is the polynomial generated by 
the first $n$ terms of the Taylor series

\end{itemize}

\end{document}
