\documentclass{article}
\usepackage[margin=0.75in]{geometry}
\usepackage{amsmath}
\usepackage{amsfonts}
\usepackage{tikz}
\usepackage{xcolor}
\usepackage{hyperref}
\usepackage{tabularx}
\usepackage{multicol}
\usepackage{graphicx}
\graphicspath{{./}}

\def\chapt{10.2}
\def\chaptname{Infinite Series}

\begin{document}

\noindent
{\scshape Math 142} \hfill {\scshape \S\chapt~ notes} \hfill {\scshape Spring 2024}

\smallskip

\hrule

\bigskip

\hfill
\\

\thispagestyle{empty}

{
\huge
\noindent
\textbf{\chapt~\chaptname}
}

\section*{Main Ideas}

\begin{itemize}

\item
A \textbf{Infinite Series} is the sum of an infinite \underline{sequence}
\begin{gather*}
S_n ~~~=~~~
\sum_{n=1}^{\infty} a_n
~~~=~~~
a_1 + a_2 + a_3 + ...
\end{gather*}

\item A \textbf{Partial Sum} is the sum of the first $n$ terms of a series
\begin{gather*}
s_n ~~~=~~~
\sum_{k=1}^n a_k
~~~=~~~
a_1 + a_2 + a_3 + ... + a_n
\end{gather*}
if the sequence of partial sums $\{~s_n~\}$ converges to $L$ as $n \to \infty$,
then we say the infinite series converges to $L$
\\

\item
The \textbf{$N$-th term test}
\begin{gather*}
\text{if}~~~
\lim_{n \to \infty} a_n \neq 0
~~~\text{then the series}~~~
\sum_{n=1}^{\infty} a_n
~~~\text{diverges}
\\
\\
\text{(otherwise the test is \underline{inconclusive})}
\\
\end{gather*}

\item
\textbf{Geometric Series}
\begin{gather*}
\text{are series in the form}~~~
\sum_{n=1}^{\infty} ar^{n-1}=a+ar+ar^2+ar^3+...
\\
\\
\text{the partial sum of the series is}~~~
s_n ~~~=~~~ a+ar+ar^2+...+ar^{n-1}~~~=~~~\frac{a(1-r^n)}{1-r}
\\
\\
\text{if }~~~ |r|<1 ~~~\text{ then the series \underline{converges} to }~~~
\frac{a}{1-r}
~~~\text{otherwise it \underline{diverges}}
\end{gather*}

\end{itemize}

\end{document}