\documentclass{article}
\usepackage[margin=0.75in]{geometry}
\usepackage{amsmath}
\usepackage{amsfonts}
\usepackage{tikz}
\usepackage{xcolor}
\usepackage{hyperref}
\usepackage{tabularx}
\usepackage{multicol}

\def\chapt{10.6}
\def\chaptname{Alternating Series, Conditional Convergence}

\begin{document}

\noindent
{\scshape Math 142} \hfill {\scshape \S\chapt~ notes} \hfill {\scshape Spring 2024}

\smallskip

\hrule

\bigskip

\hfill
\\

\thispagestyle{empty}

{
\huge
\noindent
\textbf{\chapt~\chaptname}
}

\section*{Main Ideas}

\begin{itemize}

\item
\textbf{Alternating Series Test}
\begin{gather*}
\text{the series}~~~~~~
\sum_{n=1}^{\infty} (-1)^{n+1} ~u_n = u_1 - u_2 + u_3 - u_4 + ...
~~~~~~\text{converges if and only if}
\\
\end{gather*}

\begin{enumerate}
\item
The $u_n$ terms are all positive

\item
The $u_n$ terms are eventually non-increasing ($u_{n+1} \leq u_n$ for all $n \geq N$ for some $N$)

\item
The $u_n$ terms approach 0 ($\lim u_n=0$ as $n\to \infty$)
\end{enumerate}

\item
\textbf{Conditional Convergence}
\\

If a series is \underline{convergent}, but not \underline{absolutely convergent},
then it is \underline{conditionally convergent}

\begin{gather*}
\text{if}~~~~
\sum_{n=1}^{\infty} |a_n|
~~~~\text{converges, then}~~~~
\sum_{n=1}^{\infty} a_n
~~~~\text{is absolutely convergent}
\end{gather*}

\end{itemize}
\hfill
\\

\section*{Summary of Convergence Tests}

\begin{enumerate}
\item \textbf{$N$-th term test}
\item \textbf{Geometric series}
\item \textbf{$P$-series}
\item \textbf{Integral test}
\item \textbf{Direct comparison}
\item \textbf{Limit comparison}
\item \textbf{Root test}
\item \textbf{Ratio test}
\item \textbf{Absolute convergence}
\item \textbf{Alternating series test}
\end{enumerate}

\newpage

\section*{Homework Problems}

\noindent
\textbf{1.}
\begin{gather*}
\sum_{n=1}^{\infty} (-1)^{n+1} \frac{1}{\sqrt{n}}
\\
\\
\text{1. terms are all positive since } n > 0 \implies \sqrt{n} > 0
\implies \frac{1}{\sqrt{n}} > 0 \text{~always}
\\
\text{2. terms are decreasing since } d \left[ \frac{1}{\sqrt{x}} \right]
= \frac{-1}{2x^{3/2}} < 0 \text{~always}
\\
\text{3.}
\lim_{n \to \infty} \frac{1}{\sqrt{n}}=0
\\
\\
\implies \text{conv by alternating series test}
\end{gather*}
\hfill
\\




\noindent
\textbf{3.}
\begin{gather*}
\sum_{n=1}^{\infty} (-1)^{n+1} \frac{1}{n3^n}
\\
\\
\text{1. terms are all positive since } n > 0 \implies n3^n > 0
\implies \frac{1}{n3^n}>0 \text{~always}
\\
\text{2. terms are decreasing since } d \left[ \frac{1}{x3^x} \right]
= \frac{-1}{x3^x} \cdot (3^x+x 3^x \ln 3) < 0 \text{~always}
\\
\text{3. }
\lim_{n \to \infty} \frac{1}{n3^n} = 0
\\
\\
\implies \text{conv by alternating series test}
\end{gather*}
\hfill
\\




\noindent
\textbf{5.}
\begin{gather*}
\sum_{n=1}^{\infty} (-1)^{n+1} \frac{n}{n^2+1}
\\
\\
\text{1. $u_n$ terms are all positive since top and bottom always positive when } n \geq 1
\\
\text{2. terms are decreasing since }
d \left[ \frac{x}{x^2+1} \right]
=\frac{(x^2+1)-x \cdot 2x}{(x^2+1)^2}
=\frac{1-x^2}{(x^2+1)^2}<0
\text{~when } x \geq 1
\\
\text{3. } \lim_{n \to \infty} \frac{n}{n^2+1}=0
\\
\\
\implies \text{conv by alternating series test}
\end{gather*}
\hfill
\\




\noindent
\textbf{7.}
\begin{gather*}
\sum_{n=1}^{\infty} (-1)^{n+1} \frac{2^n}{n^2}
\\
\\
\text{3. does not meet condition because}
\\
\lim_{n \to \infty} \frac{2^n}{n^2}
\xrightarrow{\text{L'Hopital's}}
\lim_{n \to \infty} \frac{2^n \cdot \ln 2}{2n}
\xrightarrow{\text{L'Hopital's}}
\lim_{n \to \infty} \frac{2^n \cdot \ln 2 \cdot \ln 2}{2}=\infty \neq 0
\\
\\
\implies \text{div by $n$-th term test}
\end{gather*}
\hfill
\\



\noindent
\textbf{11.}
\begin{gather*}
\sum_{n=1}^{\infty} (-1)^{n+1} \frac{\ln n}{n}
\\
\\
\text{1. $u_n$ terms are all positive since top and bottom always positive when } n > 1
\\
\text{2. terms are decreasing since }
d \left[ \frac{\ln x}{x^2} \right]
= \frac{x^{-1}\cdot x^2 - \ln x \cdot 2x}{(x^2)^2}
= \frac{x(1 - 2\ln x)}{(x^2)^2}>0
\text{~when } x \geq 1
\\
\text{3. } \lim_{n \to \infty} \frac{\ln n}{n^2}
\xrightarrow{\text{L'Hopital's}} \frac{1/n}{2n}=\frac{1}{2n^2}=0
\\
\\
\implies \text{conv by alternating series test}
\end{gather*}
\hfill
\\



\noindent
\textbf{15.}
\begin{gather*}
\sum_{n=1}^{\infty} (-1)^{n+1} (0.1)^n
\\
\sum_{n=1}^{\infty} \left| (-1)^{n+1} (0.1)^n \right|
= \sum_{n=1}^{\infty} (0.1)^n
\\
\\
\text{conv absolutely since geometric series with } r=0.1 \implies |r|<1
\end{gather*}
\hfill
\\



\newpage



\noindent
\textbf{17.}
\begin{gather*}
\sum_{n=1}^{\infty} (-1)^{n+1} \frac{1}{\sqrt{n}}
\\
\sum_{n=1}^{\infty} \left| (-1)^{n+1} \frac{1}{\sqrt{n}} \right|
=
\sum_{n=1}^{\infty} \frac{1}{\sqrt{n}}
\\
\\
\text{not abs conv since $p$-series div with } p=1/2<1
\\
\\
\text{alternating series test:}
\\
\\
\text{1. terms are all positive since } \sqrt{n}>0 \text{~always}
\\
\text{2. terms are decreasing since }
d \left[ \frac{1}{\sqrt{x}} \right] = \frac{-1}{2x^{3/2}} < 0 \text{~always when } x \geq 1
\\
\text{3. }
\lim_{n \to \infty} \frac{1}{\sqrt{n}} = 0
\\
\\
\implies \text{converges by alternating series test}
\implies \text{conditionally conv}
\end{gather*}
\hfill
\\




\noindent
\textbf{19.}
\begin{gather*}
\sum_{n=1}^{\infty} (-1)^{n+1} \frac{n}{n^3+1}
\\
\\
\text{compare with } b_n = \frac{1}{n^2}
\\
\lim_{n \to \infty} \left| \frac{a_n}{b_n} \right|
=\lim_{n \to \infty} \frac{n}{n^3+1} \div \frac{1}{n^2}
=\lim_{n \to \infty} \frac{n^3}{n^3+1}=1
\\
\\
\implies \text{both converge or both diverge by limit comparison}
\\
\implies \text{conv since $1/n^2$ is $p$-series with $p=2>1$}
\implies \text{abs conv}
\end{gather*}
\hfill
\\




\noindent
\textbf{21.}
\begin{gather*}
\sum_{n=1}^{\infty} (-1)^n \frac{1}{n+3}
\\
\\
\text{not abs conv since $p$-series with $p=1$}
\\
\\
\text{1. all terms are positive since } n+3 > 0 \text{~always when } n \geq 1
\\
\text{2. terms are decreasing since }
d \left[ \frac{1}{x+3} \right]
=\frac{-1}{(x+3)^2} < 0
\text{~always since } (x+3)^2>0 \text{~always when } x \geq 1
\\
\text{3. }
\lim_{n \to \infty} \frac{1}{n+3}=0
\\
\\
\implies
\text{conv by alternating series test}
\implies\text{conditionally convergent}
\end{gather*}



\noindent
\textbf{23.}
\begin{gather*}
\sum_{n=1}^{\infty} (-1)^{n+1} \frac{3+n}{5+n}
\\
\\
\lim_{n \to \infty} \frac{3+n}{5+n}=1 \neq 0 \implies \text{div by $n$-th term test}
\end{gather*}


\noindent
\textbf{31.}
\begin{gather*}
\sum_{n=1}^{\infty} (-1)^n \frac{n}{n+1}
\\
\\
\lim_{n \to \infty} \frac{n}{n+1}=1 \neq 0 \implies \text{div by $n$-th term test}
\end{gather*}


\end{document}