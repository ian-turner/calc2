\documentclass{article}
\usepackage[margin=0.75in]{geometry}
\usepackage{amsmath}
\usepackage{amsfonts}
\usepackage{tikz}
\usepackage{xcolor}
\usepackage{hyperref}
\usepackage{tabularx}
\usepackage{multicol}

\begin{document}

\noindent
{\scshape Math 142} \hfill {\scshape Exam 1 Study Guide} \hfill {\scshape Spring 2024}

\smallskip

\hrule

\bigskip

\begin{multicols}{2}
\section*{8.1 - U-substitution}

{
\large

\begin{enumerate}

\item
$\int k~dx=kx+c$

\item
$\int x^n~dx=\frac{x^{n+1}}{n+1}+C$

\item
$\int \frac{dx}{x}=\ln|x|+c$

\item
$\int e^x~dx=e^x+c$

\item
$\int a^x~dx=\frac{a^x}{\ln a}+c~(a>0,a \neq 1)$

\item
$\int \sin x~dx=-\cos x+c$

\item
$\int \cos x~dx=\sin x+c$

\item
$\int \sec^2 x~dx=\tan x+c$

\item
$\int \csc^2 x~dx=-\cot x+c$

\item
$\int \sec x \tan x~dx=\sec x+c$

\item
$\int \csc x \cot x ~dx=-\csc x + c$

\item
$ \int \tan x ~dx = \ln |\sec x| +c$

\item
$\int \cot x ~dx = \ln| \sin x| + c$

\item
$\int \sec x ~dx = \ln|\sec x + \tan x| +c$

\item
$\int \csc x ~dx = -\ln|\csc x + \cot x| + c$

\item
$\int \frac{dx}{\sqrt{a^2-x^2}}=\arcsin \left( \frac{x}{a} \right) + c$

\item
$\int \frac{dx}{a^2+x^2}=\frac{1}{a} \arctan \left| \frac{x}{a} \right| + c$

\item
$\int \frac{dx}{x\sqrt{x^2-a^2}}=\frac{1}{a} \text{arcsec} \left| \frac{x}{a} \right| + c$

\end{enumerate}
}

\section*{8.2 - Integration by parts}

\begin{center}
\huge
$\int u~dv=uv-\int v~du$
\end{center}

\section*{8.3 - Trigonometric integrals}

Use the following process for evaluating products of powers of $\sin$ and $\cos$.
The process is similar for powers of $\tan$ and $\sec$, except that you will always choose
$\tan$ to be your substitution.

{\large
\begin{center}
$\int \sin^m x \cos^n x ~dx$
\end{center}
}

\textbf{Case 1} If $m$ is odd, write $m$ as $2k+1$ and use identity $\sin^2x=1-\cos^2x$.
Then sub $u=\cos x$.

\textbf{Case 2} If $n$ is odd, write $n$ as $2k+1$ and use the identity $\cos^2 x = 1-\sin^2 x$.
Then sub $u=\sin x$.

\textbf{Case 3} If both $m$ and $n$ are even sub using half angle formula.

\section*{8.4 - Trigonometric substitutions}

\begin{center}


\tikzset{every picture/.style={line width=0.75pt}} %set default line width to 0.75pt        

\begin{tikzpicture}[x=0.75pt,y=0.75pt,yscale=-1,xscale=1]
%uncomment if require: \path (0,506); %set diagram left start at 0, and has height of 506

%Shape: Right Triangle [id:dp8253324439760634] 
\draw   (180,45.25) -- (50,140) -- (180,140) -- cycle ;
%Shape: Right Triangle [id:dp3759848084047377] 
\draw   (180,175.25) -- (50,270) -- (180,270) -- cycle ;
%Shape: Right Triangle [id:dp8101024609558598] 
\draw   (180,305.25) -- (50,400) -- (180,400) -- cycle ;

% Text Node
\draw (61,62) node [anchor=north west][inner sep=0.75pt]   [align=left] {$\displaystyle \sqrt{a^{2} +x^{2}}$};
% Text Node
\draw (191,81) node [anchor=north west][inner sep=0.75pt]   [align=left] {$\displaystyle x$};
% Text Node
\draw (111,142) node [anchor=north west][inner sep=0.75pt]   [align=left] {$\displaystyle a$};
% Text Node
\draw (82,273) node [anchor=north west][inner sep=0.75pt]   [align=left] {$\displaystyle \sqrt{a^{2} -x^{2}}$};
% Text Node
\draw (189,221) node [anchor=north west][inner sep=0.75pt]   [align=left] {$\displaystyle x$};
% Text Node
\draw (101,202) node [anchor=north west][inner sep=0.75pt]   [align=left] {$\displaystyle a$};
% Text Node
\draw (101,332) node [anchor=north west][inner sep=0.75pt]   [align=left] {$\displaystyle x$};
% Text Node
\draw (111,402) node [anchor=north west][inner sep=0.75pt]   [align=left] {$\displaystyle a$};
% Text Node
\draw (181,343) node [anchor=north west][inner sep=0.75pt]   [align=left] {$\displaystyle \sqrt{x^{2} -a^{2}}$};
% Text Node
\draw (291,82) node [anchor=north west][inner sep=0.75pt]   [align=left] {$\displaystyle x=a\tan \theta $};
% Text Node
\draw (291,222) node [anchor=north west][inner sep=0.75pt]   [align=left] {$\displaystyle x=a\sin \theta $};
% Text Node
\draw (291,351) node [anchor=north west][inner sep=0.75pt]   [align=left] {$\displaystyle x=a\sec \theta $};


\end{tikzpicture}

\end{center}


\section*{8.5 - Partial fractions}

For when you have a rational function of two polynomials

{
\huge
\begin{center}
$\frac{f(x)}{g(x)}$
\end{center}
}

\begin{enumerate}

\item
Make sure $g(x)$ has the same or higher order than $f(x)$.
The order is the number of its highest exponent term, ex: $x^2$ has order 2, $x^3+2x$ has order 3.

\item
If $f(x)$ has the higher order, do polynomial long division first.

\item
Factor $g(x)$.

\item
If $g(x)$ has a linear factor $(x-r)^n$, then the fraction decomposition will contain

\begin{equation*}
\frac{A_1}{(x-r)}+
\frac{A_2}{(x-r)^2}+
...+
\frac{A_n}{(x-r)^n}
\end{equation*}

\item
If $g(x)$ has an unfactorable quadratic factor $(ax^2+bx+c)^n$ (it will have complex roots), then
the decomposition will contain

\begin{equation*}
\frac{A_1 x+B_1}{(ax^2+bx+c)}+
\frac{A_2 x+B_2}{(ax^2+bx+c)^2}+
...+
\frac{A_n x+B_n}{(ax^2+bx+c)^n}
\end{equation*}

\end{enumerate}

\section*{8.8 - Improper integrals}

For when you have:

\begin{enumerate}

\item
an infinity in the limit, like
$\int_1^{\infty} \frac{dx}{\sqrt{x}}$.

\item
an invalid input in your bounds, like
$\int_0^1 \frac{dx}{x}$.

\item
a discontinuity or asymptote within the integration region, like
$\int_{-1}^1 \frac{dx}{x^{2/3}}$.

\end{enumerate}

\noindent
Situation 1 and 2 can be dealt with by adding a parameter and a limit:

\begin{equation*}
\int_1^{\infty} \frac{dx}{x}
=\lim_{t \to \infty} \int_1^t \frac{dx}{x}
\end{equation*}
\\

\noindent
And situation 3 is the same, except you will end up with two limits from either side:

\begin{equation*}
\int_{-1}^1 \frac{dx}{x^{2/3}}
=\lim_{t \to 0^-} \int_{-1}^t \frac{dx}{x^{2/3}}
+\lim_{t \to 0^+} \int_{t}^1 \frac{dx}{x^{2/3}}
\end{equation*}

\section*{Trigonometric identities}

{\large
\begin{itemize}

\item
$\sin^2 x + \cos^2 x = 1$

\item
$\tan^2 x + 1 = \sec^2 x$

\item
$\sin \theta = \pm \sqrt{\frac{1-\cos 2\theta}{2}}$

\item
$\cos \theta = \pm \sqrt{\frac{1+\cos 2\theta}{2}}$

\item
$\sin 2\theta = \sin \theta \cos \theta$

\item
$\sin \theta \sin \phi = \frac{ \cos(\theta-\phi)-\cos(\theta+\phi) }{2}$

\item
$\cos \theta \cos \phi = \frac{ \cos(\theta+\phi)+\cos(\theta-\phi) }{2}$

\item
$\sin \theta \cos \phi = \frac{ \sin(\theta+\phi)+\sin(\theta-\phi) }{2}$

\item
$\cos \theta \sin \phi = \frac{ \sin(\theta+\phi)-\sin(\theta-\phi) }{2}$

\end{itemize}
}

\section*{Polynomial long division}

To divide $f(x)$ by $g(x)$:

\begin{enumerate}

\item
Place $f(x)$ inside the division symbol

\item
Find what you need to multiply $f(x)$ by to cancel out the
highest order term in $g(x)$.

\item
Multiply $f(x)$ by this and subtract from $g(x)$

\item
Write the new function bellow

\item
Repeat until the order of the resulting
function is less than the order of $f(x)$

\end{enumerate}

\noindent
\textbf{
Example:
}

$x^2+2x+3$ divided by $x-2$:

\begin{equation*}
\begin{array}{r}
x+4~~~~~~~~\phantom{)}   \\
x-2{\overline{\smash{\big)}\,x^2+2x+3\phantom{)}}}\\
\underline{-~\phantom{(}(x^2-2x)\phantom{-b)}}\\
0+4x+3\phantom{)}\\ 
\underline{-~\phantom{()}(4x-8)}\\ 
0+11\phantom{)}
\end{array}
\\
=(x+4)+\frac{11}{x^2+2x+3}
\end{equation*}

\section*{L'Hopital's rule}

For when you are evaluating a limit in the form:

\begin{equation*}
\lim_{t \to a} \frac{f(x)}{g(x)}
=\frac{0}{0}
~~\text{OR}~~
\frac{\infty}{\infty}
\end{equation*}

\noindent
You can take the derivative of both parts:

\begin{equation*}
\lim_{t \to a} \frac{f(x)}{g(x)}
=\lim_{t \to a} \frac{f'(x)}{g'(x)}
\end{equation*}

\noindent
And keep taking derivatives until you reach something easy
to take the limit of.

\end{multicols}

\end{document}